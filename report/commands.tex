%Packages
\usepackage{geometry}
\usepackage{amsmath}
\usepackage{amsfonts}
\usepackage{mathtools}
\usepackage{amsthm}
\usepackage{mhchem}
\usepackage{multicol}
\usepackage{gensymb}
\usepackage{chemfig}
\usepackage{tikz}
\usepackage{pgfplots}
\usepackage{listings}
\usepackage{xcolor}
\usepackage{hyperref}
\usepackage{booktabs}
\usepackage{pgfplotstable}
\usepackage{graphicx}
\usepackage{caption}
\usepackage{commath}
\usepackage{centernot}
\usepackage{clrscode3e}
\usepackage{tikz-cd}
\usepackage{enumitem}
\usepackage{cryptocode}
\usepackage{amssymb}
\usepackage[shortlabels]{enumitem}
\hypersetup{
    colorlinks,
    citecolor=black,
    filecolor=black,
    linkcolor=black,
    urlcolor=black
}
\graphicspath{ {./images/} }
%Custom Environments

%A generic custom theorem header
\newtheoremstyle{ntheoremstyle}% name of the style to be used
{}% measure of space to leave above the theorem. E.g.: 3pt
{}% measure of space to leave below the theorem. E.g.: 3pt
{\it}% name of font to use in the body of the theorem
{}% measure of space to indent
{\bfseries}% name of head font
{}% punctuation between head and body
{ }% space after theorem head; " " = normal interword space
{\thmname{#3}.}
\theoremstyle{ntheoremstyle}
\newtheorem{ntheorem}{Theorem}[section]

\newtheoremstyle{ndefinitionstyle}% name of the style to be used
{}% measure of space to leave above the theorem. E.g.: 3pt
{}% measure of space to leave below the theorem. E.g.: 3pt
{}% name of font to use in the body of the theorem
{}% measure of space to indent
{\bfseries}% name of head font
{}% punctuation between head and body
{ }% space after theorem head; " " = normal interword space
{\thmname{#3}.}
\theoremstyle{ndefinitionstyle}
\newtheorem{ndefinition}{Theorem}[section]

\newtheoremstyle{nproofstyle}% name of the style to be used
{}% measure of space to leave above the theorem. E.g.: 3pt
{}% measure of space to leave below the theorem. E.g.: 3pt
{}% name of font to use in the body of the theorem
{}% measure of space to indent
{\bfseries}% name of head font
{}% punctuation between head and body
{ }% space after theorem head; " " = normal interword space
{\thmname{#3}.}
\theoremstyle{nproofstyle}
\newtheorem{nproof}{Proof}[section]
%Custom Commands
\newcommand{\ringmod}[1]{$\mathbb{Z}$/#1$\mathbb{Z}$}
\newcommand{\Z}{\mathbb{Z}}
\newcommand{\R}{\mathbb{R}}
\newcommand{\N}{\mathbb{N}}
\newcommand{\C}{\mathbb{C}}
\newcommand{\F}{\mathbb{F}}
\newcommand{\Q}{\mathbb{Q}}
\newcommand{\Eu}{\mathbb{E}}
\newcommand{\PGL}{\text{PGL}}
\newcommand{\GL}{\text{GL}}
\newcommand{\Aut}{\text{Aut}}
\newcommand{\Endo}{\text{End}}
\newcommand{\E}[1]{\mathbb{E}\left[ #1 \right]}
\newcommand{\Mod}[1]{\ (\mathrm{mod}\ #1)}
\newcommand{\subproblem}[2]{\noindent \ \ \ \ \ \ \ \ \ \ (#1) \ \ #2 \newline\noindent}
\newcommand{\Lim}{\lim\limits}
\newcommand{\Int}{\displaystyle \int}
\newcommand{\Equation}[2]{\begin{equation}
    #2 \tag{#1} \label{eq: #1}
\end{equation}}
\newcommand{\trace}{\text{trace} \ }
\newcommand{\Var}{\text{Var}}
\newcommand{\Tor}{\text{Tor}}
\newcommand{\cross}{\times}
\newcommand{\vmat}[1]{\begin{vmatrix} #1 \end{vmatrix}}
\newcommand{\bmat}[1]{\begin{bmatrix} #1 \end{bmatrix}}
\newcommand{\pmat}[1]{\begin{pmatrix} #1 \end{pmatrix}}
\newcommand{\eqq}{\stackrel{?}{=}}
\newcommand{\p}[1]{\left( #1 \right)}
\DeclarePairedDelimiter\floor{\lfloor}{\rfloor}
\DeclarePairedDelimiter\ceil{\lceil}{\rceil}
\DeclarePairedDelimiter\babs{\bigg \lvert}{\bigg \rvert}
\newcommand{\note}[1]{\begin{changemargin}{.5cm}{.5cm}\emph{\underline{Note}:} #1 \end{changemargin}}
\def\changemargin#1#2{\list{}{\rightmargin#2\leftmargin#1}\item[]}
\let\endchangemargin=\endlist 
\newcommand{\xor}{\ \underline{\lor} \ }
\newcommand{\suchthat}{\ \text{s.t.} \ }
\newcommand{\spanof}{\text{span} \ }
\newcommand{\code}[1]{\lstinline{#1}}
\DeclareMathOperator*{\argmax}{arg\,max}
\DeclareMathOperator*{\argmin}{arg\,min}
\newcommand{\tr}{\text{tr} \ }
\newcommand{\sign}{\text{sign}}
\newcommand{\mc}[1]{\mathcal{#1}}
%Crypto Stuff
\newcommand{\sendright}[1]{\sendmessageright{top=\text{#1}}}
\newcommand{\sendleft}[1]{\sendmessageleft{top=\text{#1}}}

%Chem Stuff
\newcommand{\aq}{_{(aq)}}

\newcommand{\rarrow}{$\to$ }
\newcommand{\imp}{$\implies$}
%Physics
\newcommand{\ihat}{\hat{i}}
\newcommand{\jhat}{\hat{j}}
\newcommand{\khat}{\hat{k}}
\newcommand{\avg}{\text{avg}}
\usepackage{mathtools}
\DeclarePairedDelimiter\bra{\langle}{\rvert}
\DeclarePairedDelimiter\bk{\langle}{\rangle}
\DeclarePairedDelimiter\ket{\lvert}{\rangle}
\DeclarePairedDelimiterX\braket[2]{\langle}{\rangle}{#1 \delimsize\vert #2}
%Units
\newcommand{\m}{\text{m}}
\newcommand{\nm}{\text{nm}}
\newcommand{\s}{\text{s}}
\newcommand{\kg}{\text{kg}}
\newcommand{\ft}{\text{ft}}
\newcommand{\g}{\text{g}}
\newcommand{\J}{\text{J}}
\newcommand{\mol}{\text{mol}}
\newcommand{\dC}{\degree \text{C}}
\newcommand{\dK}{\degree \text{K}}
\newcommand{\Ne}{\text{N}}
\newcommand{\net}{\text{net}}
\newcommand{\ang}{\mbox{\normalfont\AA}}
\newcommand{\atm}{\text{atm}}
\newcommand{\Co}{\text{C}}
\newcommand{\A}{\text{A}}
\newcommand{\V}{\text{V}}
\newcommand{\K}{\text{K}}
\newcommand{\q}[1]{\textbf{#1}}
\newcommand{\rank}{\text{rank}}
\newcommand{\ran}{\text{Ran}}
\newcommand{\detailtexcount}[1]{%
  \immediate\write18{texcount -merge -sum -incbib -dir #1.tex > #1.wcdetail }%
  \verbatiminput{#1.wcdetail}%
}
\newcommand{\Res}{\text{Res}}
\newcommand{\Gal}{\text{Gal}}
\newcommand{\tensor}{\otimes}
\newcommand{\im}{\text{Im} \ }
\newcommand{\sinc}{\text{sinc} \ }
\newcommand{\inner}[2]{\langle #1, #2 \rangle}

\newcommand{\from}{\leftarrow}

\usetikzlibrary{arrows}

\newcommand{\quickwordcount}[1]{%
  \immediate\write18{texcount -1 -sum=1,0,0,0,0,0,0. -merge #1.tex > #1-words.sum }%
  \input{#1-words.sum}%
}

\newcommand{\quickcharcount}[1]{%
  \immediate\write18{texcount -1 -sum -merge -char #1.tex > #1-chars.sum }%
  \input{#1-chars.sum}%
}

\newcommand{\bibent}{\noindent \hangindent 40pt}
\newenvironment{workscited}{\newpage \begin{center} Works Cited \end{center}}{\newpage }

%New colors defined below
\definecolor{codegreen}{rgb}{0,0.6,0}
\definecolor{codegray}{rgb}{0.5,0.5,0.5}
\definecolor{codepurple}{rgb}{0.58,0,0.82}
\definecolor{backcolour}{rgb}{0.95,0.95,0.92}

%Code listing style named "mystyle"
\lstdefinestyle{mystyle}{
  backgroundcolor=\color{backcolour},   commentstyle=\color{codegreen},
  keywordstyle=\color{magenta},
  numberstyle=\tiny\color{codegray},
  stringstyle=\color{codepurple},
  basicstyle=\ttfamily\footnotesize,
  breakatwhitespace=false,         
  breaklines=true,                 
  captionpos=b,                    
  keepspaces=true,                 
  numbers=left,                    
  numbersep=2pt,                  
  showspaces=false,                
  showstringspaces=false,
  showtabs=false,                  
  tabsize=2
}
%"mystyle" code listing set
\lstset{style=mystyle}

\newcommand{\lstnextline}[1]{\lstdefinestyle{mystyle#1}{
  backgroundcolor=\color{backcolour},   commentstyle=\color{codegreen},
  keywordstyle=\color{magenta},
  numberstyle=\tiny\color{codegray},
  stringstyle=\color{codepurple},
  basicstyle=\ttfamily\footnotesize,
  breakatwhitespace=false,         
  breaklines=true,                 
  captionpos=b,                    
  keepspaces=true,                 
  numbers=left,                    
  numbersep=5pt,                  
  showspaces=false,                
  showstringspaces=false,
  showtabs=false,                  
  tabsize=2,
  firstnumber=#1
}
%"mystyle" code listing set
\lstset{style=mystyle#1}}

\usepackage{calligra}

\DeclareMathAlphabet{\mathcalligra}{T1}{calligra}{m}{n}
\DeclareFontShape{T1}{calligra}{m}{n}{<->s*[2.2]callig15}{}
\newcommand{\gr}{\mathcalligra{r}\,}
\newcommand{\bgr}{\pmb{\mathcalligra{r}}\,}

\setlength\parindent{0pt}